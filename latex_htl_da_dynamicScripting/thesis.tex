% !TEX root = thesis.tex
\documentclass[12pt,a4paper,titlepage,listof=totoc,bibliography=totoc,chapteratlists=0pt]{scrreprt}
\usepackage{amsfonts, amssymb, amsmath, verbatim, float}
\usepackage{listings}

\lstdefinestyle{yaml}{
     basicstyle=\color{blue}\footnotesize,
     rulecolor=\color{black},
     string=[s]{'}{'},
     stringstyle=\color{blue},
     comment=[l]{:},
     commentstyle=\color{black},
     morecomment=[l]{-}
 }

 \lstdefinelanguage{JavaScript}{
  morekeywords=[1]{break, continue, delete, else, for, function, if, in,
    new, return, this, typeof, var, void, while, with},
  % Literals, primitive types, and reference types.
  morekeywords=[2]{false, null, true, boolean, number, undefined,
    Array, Boolean, Date, Math, Number, String, Object},
  % Built-ins.
  morekeywords=[3]{eval, parseInt, parseFloat, escape, unescape},
  sensitive,
  morecomment=[s]{/*}{*/},
  morecomment=[l]//,
  morecomment=[s]{/**}{*/}, % JavaDoc style comments
  morestring=[b]',
  morestring=[b]"
}[keywords, comments, strings]
\input{header}

\makeatletter
\def\bstctlcite{\@ifnextchar[{\@bstctlcite}{\@bstctlcite[@auxout]}}
\def\@bstctlcite[#1]#2{\@bsphack
	\@for\@citeb:=#2\do{%
		\edef\@citeb{\expandafter\@firstofone\@citeb}%
		\if@filesw\immediate\write\csname #1\endcsname{\string\citation{\@citeb}}\fi}%
	\@esphack}
\makeatother

\clubpenalty=10000 
\widowpenalty=10000
\displaywidowpenalty=10000
\interfootnotelinepenalty=10000

\title{Dynamische Anpassung von .Net Anwendungen zur Laufzeit}
\author{Robert Freiseisen, Phillip Füreder}

\makeindex
\makeglossaries
\begin{document}
\bstctlcite{IEEEexample:BSTcontrol}
\newcommand{\reminder}[1]
{ \textcolor{red}{<[{\bf\marginpar{\mbox{$<==$}} #1 }]>} }
\newcommand{\icode}[1]{\lstinline$#1$}
%\urlstyle{same}
%\setstretch{1.5}
\setstretch {1.433}
\renewcommand{\arraystretch}{1.2}

\includepdf{./titlepage/coversheet}
\pagenumbering{Roman}
\newpage
\input{oath}
\begin{spacing}{1}
    \chapter*{Abstract}
\end{spacing}
\begin{wrapfigure}{r}{0.3\textwidth}
    \begin{center}
      \includegraphics[width=0.3\textwidth]{pics/coding.jpg}
    \end{center}
\end{wrapfigure}
This diploma thesis is dedicated to the extension of standard software through the targeted use of 
scripting. The focus is on the development of flexible mechanisms that make it possible to adapt 
and extend existing software to individual requirements using scripts. The work examines the 
applicability of this method in various contexts and evaluates its effectiveness and adaptability.
\\\\
We demonstrate the practical implementation of these concepts using an example application. The 
development of this application demonstrates not only the theoretical feasibility but also the 
practical suitability of the approach. The investigation focuses on the efficiency and scalability 
of the scripting solutions, analyzing resource consumption and runtime to enable a well-founded 
assessment of their performance.

\newpage
\begin{spacing}{1}
    \chapter*{Zusammenfassung}
\end{spacing}
\begin{wrapfigure}{r}{0.3\textwidth}
    \begin{center}
      \includegraphics[width=0.3\textwidth]{pics/coding.jpg}
    \end{center}
\end{wrapfigure}
Diese Diplomarbeit widmet sich der Erweiterung von Standardsoftware durch den gezielten Einsatz 
von Scripting. Der Fokus liegt auf der Entwicklung flexibler Mechanismen, die es ermöglichen, 
bestehende Software durch den Einsatz von Skripten an individuelle Anforderungen anzupassen und 
zu erweitern. Die Arbeit untersucht die Anwendbarkeit dieser Methode in verschiedenen Kontexten 
und evaluiert ihre Wirksamkeit sowie Anpassungsfähigkeit.
\\\\
Die praktische Umsetzung dieser Konzepte zeigen wir anhand einer Beispielanwendung. Durch die 
Entwicklung dieser Anwendung wird nicht nur die theoretische Machbarkeit, sondern auch die 
Praxistauglichkeit des Ansatzes demonstriert. Die Untersuchung konzentriert sich auf die Effizienz 
und Skalierbarkeit der Scripting-Lösungen, analysiert den Ressourcenverbrauch und die Laufzeit, 
um eine fundierte Bewertung ihrer Performance zu ermöglichen.


\pagestyle{plain}

\renewcommand{\lstlistlistingname}{Quellcodeverzeichnis}

\tableofcontents
\newpage
\setcounter{RPages}{\value{page}}
\setcounter{page}{0}
\pagenumbering{arabic}
\pagestyle{scrheadings}

\begin{spacing}{1}
\chapter{Einleitung}\label{chapter:introduction}
\end{spacing}
\section {Standardsoftware vs Individualsoftware}
\setauthor{Philipp Füreder}

In der heutigen digitalisierten Welt ist Software enorm wichtig für den Erfolg von Unternehmen 
oder auch für den privaten Gebrauch.
Es gibt zwei Hauptarten von Software, die Unternehmen nutzen: \textbf{Standardsoftware} und 
\textbf{Individualsoftware}.
Beide haben Vor- und Nachteile, und es ist wichtig, 
diese gut zu verstehen, um die richtige Wahl für das eigene Unternehmen zu treffen.

\subsection*{Standardsoftware}

Standardsoftware ist üblicherweise sehr Benutzerfreundlich und einfach zu verwenden. 
Sie wird meist von großen Unternehmen entwickelt 
(Beispiel: Microsoft Office, Adobe Creative Cloud oder auch Google Workspace) 
und ist in der Anschaffung billiger als Individualsoftware. 
Der Grund dafür ist, dass Standardsoftware für die breite Masse entwickelt wird 
und für jeden Nutzer die gleichen Funktionen besitzt. 
Außerdem sind sie in der Regel weiter verbreitet und leichter zu erwerben.

Ein Nachteil von Standardsoftware besteht darin, dass sie nicht immer genau 
den individuellen Anforderungen eines Unternehmens gerecht wird. 
Es könnte vorkommen, dass spezifische Funktionen fehlen oder die Software 
nicht optimal auf die Arbeitsprozesse des Unternehmens zugeschnitten ist. 

\subsection*{Individualsoftware}

Auf der anderen Seite wird Individualsoftware speziell für ein einzelnes 
Unternehmen oder eine bestimmte Aufgabe entworfen. 
Das bedeutet, es ist eine individuelle Lösung, die exakt auf 
die Anforderungen und Bedürfnisse der Benutzer/innen zugeschnitten ist.

Hingegen hat die Verwendung von Individualsoftware auch ihre Nachteile. 
Zum einen ist sie in der Regel kostenintensiver, da die individuelle Entwicklung 
mehr Ressourcen erfordert. Zum anderen gestaltet sich die Umsetzung und Wartung 
anspruchsvoller, da die Einzigartigkeit der Software spezifische Herausforderungen 
mit sich bringt, die über die Zeit hinweg bewältigt werden müssen.



\begin{spacing}{1}
\chapter{Evaluierung von Skriptsprachen}
\end{spacing}
\section{Auswahl der Skriptsprachen}
\setauthor{Robert Freiseisen}

Die Wahl der Scriptsprachen wurde auf Grund von Internet-Recherchen 
und Fachgesprächen mit Mitschüler*innen und Professor*innen gefällt.\\
Folgende Scriptsprachen wurden gewählt:

\begin{itemize}
    \item Lua
    \item IronPython
    \item C\#script
    \item Javascript
\end{itemize}

\subsection{Lua}
Lua ist eine Programmiersprache, die für ihre hohe Ausführungsgeschwindigkeit geschätzt wird. 
Diese Schnelligkeit, kombiniert mit dem geringen Speicherbedarf der Sprache, macht sie ideal für ressourcenbeschränkte Umgebungen und eingebettete Systeme. 
Lua bietet eine "von Haus aus"-Nutzbarkeit, die es Entwicklern ermöglicht, sofort nach der Installation loszulegen, ohne sich um eine Vielzahl von Abhängigkeiten kümmern zu müssen. 
Ihre Einbettbarkeit ist ein weiteres Kernelement, das sie besonders attraktiv für Softwareprojekte macht, die eine integrierte Skriptsprache benötigen. Besonders in der Spieleindustrie hat Lua sich als beliebte Wahl für das Scripting etabliert. Hier ermöglicht es Entwicklern, schnell interaktive und flexible Spielmechanismen zu implementieren, ohne die Hauptspiellogik zu beeinträchtigen. 
Als Open-Source-Software steht Lua zudem einer breiten Entwicklergemeinschaft zur Verfügung, die zur kontinuierlichen Verbesserung und Erweiterung der Sprache beiträgt.
All diese Aspekte machen Lua zu einer vielseitigen Option für eine Reihe von Anwendungen, insbesondere für das Scripting in Videospielen, wie in "World of Warcraft" oder "Roblox".
\cite{gameScriptingMastery} \cite{luaDocs} \cite{programingInLua} \cite{nluaWebside} 

\newpage
\subsection{IronPython}
IronPython ist eine Open-Source-Implementierung der Programmiersprache Python, die auf der .NET-Plattform läuft. Es wurde ursprünglich von Jim Hugunin entwickelt und ist eng mit Microsoft assoziiert. IronPython ist in C\# geschrieben und ermöglicht die nahtlose Integration von Python-Code mit .NET-Anwendungen. Mit IronPython können Entwickler sowohl auf .NET-Bibliotheken als auch auf Python-Bibliotheken zugreifen, wodurch eine erweiterte Interoperabilität und Flexibilität erreicht wird.
IronPython unterstützt sowohl dynamische als auch statische Sprachfunktionen, und es ist durch die Common Language Runtime (CLR) von Microsoft vollständig integriert. Dies ermöglicht eine effiziente Ausführung von Python-Code auf der .NET-Plattform und erleichtert die Erstellung gemischter Anwendungen, die sowohl Python- als auch .NET-Code nutzen.
Beim Speicherverbrauch ist IronPython in der Regel nicht so effizient wie CPython, da das .NET mehr Overhead haben kann. 
Durch die Nutzung der Dynamic Language Runtime (DLR) von .NET kann IronPython dynamische Typen und späte Bindungen effizienter verwalten als einige andere Implementierungen, wie CPython. 
Dies ermöglicht eine enge Integration mit .NET-Bibliotheken und erleichtert die schnelle Entwicklung und Iteration von Code.
\cite{ironPython} \cite{ironPythonGithub} \cite{ironPythonInAction}
 
\newpage
\subsection{C\#script}
C\#Script ist ein Bestandteil der .NET-Plattform. Es ermöglicht die dynamische Ausführung von C\#-Code und ist in den .NET-Bibliotheken und -Laufzeitumgebungen enthalten. 
Einer der großen Vorteile von C\#Script ist, dass es sowohl in gehosteten als auch in eigenständigen Ausführungsmodellen eingesetzt werden kann. 
In einem gehosteten Modell kann das Skript innerhalb einer bestehenden Anwendung laufen und Objekte und Funktionen der Anwendung nutzen oder modifizieren. 
In einem eigenständigen Modell kann das Skript als unabhängige Anwendung ausgeführt werden. 
Ein weiteres wichtiges Merkmal ist die Kompatibilität mit .NET 5/Core und höheren Versionen. 
Dies ermöglicht eine bessere Leistung, erweiterte APIs und die Möglichkeit, plattformübergreifende Anwendungen zu entwickeln. 
Die Bibliothek stellt eine  Schnittstelle bereit, die die Integration von C\#-Skripten in eine Vielzahl von Anwendungsdomänen vereinfacht.

\cite{csharpScriptingArcticel}

\newpage 
\subsection{Javascript}
JavaScript ist eine populäre und vielseitige Programmiersprache, die vor allem in der Webentwicklung eingesetzt wird. 
Sie zeichnet sich durch ihre Einfachheit und Benutzerfreundlichkeit aus, was sie besonders für Einsteiger attraktiv macht. 
Die Sprache ist intuitiv aufgebaut, sodass die grundlegenden Konzepte schnell verstanden und angewendet werden können. 
Ein weiterer Vorteil ist, dass alle modernen Webbrowser eine JavaScript-Engine integriert haben. 
Dies ermöglicht es Entwicklern, Code unmittelbar im Browser auszuführen und zu testen, ohne zusätzliche Software installieren zu müssen. 
In Bezug auf die Sicherheit bietet JavaScript zwar einige Mechanismen, wie die Ausführung in einer Sandbox-Umgebung, um den Zugriff auf das Betriebssystem des Nutzers zu beschränken.  
Insgesamt bietet JavaScript eine ausgewogene Mischung aus Benutzerfreundlichkeit, Intuitivität und Funktionalität, allerdings müssen Entwickler stets wachsam in Bezug auf Sicherheitsrisiken sein.

Als .NET-Bibliothek wurde sich für "JavaScriptEngineSwitcher" mit "Jint" entschieden.
"JavaScriptEngineSwitcher" ist eine .NET-Bibliothek, die es ermöglicht, verschiedene JavaScript-Engines zu verwenden, und "Jint" ist eine JavaScript-Engine für .NET.


\cite{mDNWebDocs} \cite{owasp}

\newpage
\section{Kriterienkatalog}
\setauthor{Robert Freiseisen}
Jede der untersuchten Sprachen hat ihre eigenen Vor- und Nachteile, und die Auswahl der besten Option kann je nach Projektanforderungen variieren. 
In diesem Abschnitt betrachten wir diese vier Sprachen im Kontext von .NET unter verschiedenen Aspekten:

\begin{itemize}
    \item Aktivität der Entwicklung: 
    \begin{itemize}
        \item Wie lebendig und aktiv ist die Community hinter der Sprache? 
        \item Wie oft werden Updates veröffentlicht, und wie gut ist die Dokumentation?
    \end{itemize}
    \item Funktionalität:
    \begin{itemize}
        \item Welche Features bietet die Sprache, und wie reichhaltig ist ihr Ökosystem? 
        \item Gibt es umfangreiche Bibliotheken, die die Entwicklung erleichtern?
    \end{itemize}
    \item Einsetzbarkeit:
    \begin{itemize}
        \item Wie einfach lässt sich die Sprache in bestehende oder neue .NET-Projekte integrieren?
        \item Welche Voraussetzungen müssen erfüllt sein, und wie komplex ist die Integration?
    \end{itemize}
    \item Performance:
    \begin{itemize}
        \item Wie steht es um die Laufzeit-Performance des Codes? 
        \item  Inwiefern beeinflusst die Wahl der Sprache die Geschwindigkeit und Ressourceneffizienz der fertigen Anwendung?
    \end{itemize}
    \item Debugging:
    \begin{itemize}
        \item Welche Möglichkeiten bietet die Sprache für das Debugging von Code?
        \item Wie effektiv lassen sich Fehler finden und beheben, und welche Werkzeuge stehen zur Verfügung?
    \end{itemize}
\end{itemize}

\newpage
Alle Daten wurden auf zwei unterschiedlichen Geräten gemessen, da somit die Messungen gleichzeitig durchgeführt werden konnten.
Weil zwei verschiedene PCs für einen Vergleich genutzt wurden, kann dies die Vergleichbarkeit der Ergebnisse beeinflussen.

Folgende Gründe können dabei ein Faktor sein:

\begin{itemize}
    \item Hardware-Unterschiede: 
    \begin{itemize}
        \item Verschiedene PCs können erhebliche Hardware-Unterschiede aufweisen, darunter Prozessorgeschwindigkeit, Arbeitsspeicher, Grafikkarte, Festplattenkapazität und andere technische Spezifikationen. Diese Unterschiede können die Leistungsfähigkeit und Geschwindigkeit der beiden PCs stark beeinflussen.
    \end{itemize}
    \item  Betriebssystem und Software:
    \begin{itemize}
        \item Verschiedene PCs können unterschiedliche Betriebssysteme (z.B. Windows, macOS, Linux) und Softwarekonfigurationen aufweisen. Dies kann die Vergleichbarkeit der Ergebnisse beeinflussen, da bestimmte Anwendungen oder Aufgaben auf unterschiedlichen Betriebssystemen möglicherweise unterschiedlich ausgeführt werden.
    \end{itemize}
    \item Treiber und Updates:
    \begin{itemize}
        \item Die Installation und Aktualisierung von Treibern und Software kann zwischen verschiedenen PCs variieren. Dies kann zu unterschiedlichem Verhalten von Hardwarekomponenten und zur Auswirkung auf die Leistung führen.
    \end{itemize}
    \item Alter und Verschleiß:
    \begin{itemize}
        \item Ältere PCs könnten aufgrund von Abnutzung und Alterungsprozessen möglicherweise nicht mehr die gleiche Leistung erbringen wie neuere Modelle. Dies kann zu Unterschieden in der Vergleichbarkeit der Ergebnisse führen.
    \end{itemize}
    \item Systemkonfiguration: 
    \begin{itemize}
        \item Die Konfiguration von Betriebssystemeinstellungen, Energieeinstellungen und Hintergrundanwendungen kann zwischen den beiden PCs variieren, was sich auf die Ressourcennutzung und somit auf die Vergleichbarkeit auswirken kann.
    \end{itemize}
\end{itemize}

Die Daten für IronPython und Lua wurden auf den PC von Robert Freiseisen unter folgenden Voraussetzungen  gemessen:
\begin{itemize}
    \item Gerätspezifikationen:
    \begin{table}[H]
        \center
        \begin{tabular}{|p{3cm}|p{4cm}|}
            \hline
            Hersteller & HP \\ \hline
            Gerätname & LAPTOP-5U1879KR \\ \hline
            Prozessor & Intel(R) Core(TM) i5-8265U CPU @ 1.60GHz   1.80 GHz \\ \hline
            Installierter RAM & 8.00 GB (7.89 GB verwendbar) \\ \hline
            Produkt-ID & 00325-81357-65742-AAOEM \\ \hline
            Systemtyp & 64-Bit-Betriebssystem, x64-basierter Prozessor \\ \hline
        \end{tabular}
    \end{table}
    \item Betriebssystem:
    \begin{table}[H]
        \center
        \begin{tabular}{|p{4cm}|p{4cm}|}
            \hline
            Edition & Windows 11 Home \\ \hline
            Version & 21H2 \\ \hline
            Installiert am & 24.11.2021 \\ \hline
            Betriebssystembuild & 220.001.098 \\ \hline
            Leistung & Windows Feature Experience Pack 1000.22000.1098.0 \\ \hline
        \end{tabular}        
    \end{table}
\end{itemize}

\newpage
Die Daten für C\#script und Javascript wurden auf den PC von Philipp
PC B unter folgenden Voraussetzungen  gemessen:
\begin{itemize}
    \item Gerätspezifikationen:
    \begin{table}[H]
        \center
        \begin{tabular}{|p{3cm}|p{3cm}|}
            \hline
            Hersteller & Selber Gebaut \\ \hline
            Gerätname & PC-Philipp \\ \hline
            Prozessor & AMD Ryzen 5 1600 Six-Core Processor 3.20 GHz \\ \hline
            Installierter RAM & 16.00 GB \\ \hline
            Produkt-ID & 00330-80000-00000-AA542 \\ \hline
            Systemtyp & 64-Bit-Betriebssystem, x64-basierter Prozessor \\ \hline
        \end{tabular}
    \end{table}
    \item Betriebssystem:
    \begin{table}[H]
        \center
        \begin{tabular}{|p{4cm}|p{4cm}|}
            \hline
            Edition & Windows 10 Pro \\ \hline
            Version & 22H2 \\ \hline
            Installiert am & 17.08.2020 \\ \hline
            Betriebssystembuild & 19045.3393 \\ \hline
            Leistung & Windows Feature Experience Pack 1000.19044.1000.0 \\ \hline
        \end{tabular}        
    \end{table}
\end{itemize}


\newpage
\subsection{Aktivität der Entwicklung}

In der nachfolgenden Tabelle wird dargestellt, wie intensiv die Entwicklerteams an den unterschiedlichen Scriptsprachen arbeiten und ob diese auch die Nuget-Pakete entwickeln.
Es ist zu beachten, dass die Daten am 30.11.2022 erfasst wurden.
\begin{table}[H]
    \begin{tabular}{|p{3cm}|p{3cm}|p{3cm}|p{3cm}|p{3cm}|}
        \hline
        Aktivität & IronPython & Lua & C\#Scripting & Javascript\\ \hline
        Commits in den letzten 10 Monaten & 179 & 27 & 702 & 1153 \\ \hline
        Nuget-Packages vom Sprach-entwicklerteam selber &Nein &Nein &Ja & Nein\\ \hline
        Releases in den letzten 10 Monaten & 2 (2.7.1 und 3.4.0-beta1) & 1 (v5.4.4) & v4.0.0 und höher & v4.6.2 (und höher)\\ \hline
        Unterstützt aktuelle major Versionen von Scriptsprache & Ja & Ja & Ja & Ja\\ \hline
        Unterstützt aktuelle Versionen von .NET & Ja & Ja & Ja & Ja \\ \hline
    \end{tabular}
\end{table}

\subsection{Einsetzbarkeit}
Alle untersuchten Bibliotheken sind auf Windows, MAC und Linux lauffähig.

\newpage
\subsection{Performance}
Der Speicherplatz auf dem Laufwerk wurde aus dem Windows-File-Explorer entnommen. 
Die Geschwindigkeitsmessungen erfolgten mit BenchmarkDotNet. 
BenchmarkDotNet wird im Abschnitt \hyperref[sec:tech]{Verwendete Technologien} näher erleutert.

IronPython und NLua müssen externe Runtimes (für Python bzw. Lua) in die .NET-Umgebung integrieren, was zu zusätzlichem Overhead und Speicherbedarf führt. 
Im Gegensatz dazu verwenden C\#Script und JavaScriptEngineSwitcher die bereits in .NET integrierte C\#-Sprache bzw. die .NET Common Language Runtime, was einen effizienteren Speicherverbrauch ermöglicht.
 
\begin{table}[H] 
    \begin{tabular}{|p{3cm}|p{3cm}|p{3cm}|p{3cm}|p{3cm}|}
        \hline
        Performance & IronPython & Lua & C\#Scripting & Javascript\\ \hline
        Speicher einer einfachen Anwendung & 13 MB & 7.3 MB & 168 KB & 416 KB  \\ \hline
        Durchnittliche Laufzeit einer einfachen Anwendung & 137.118 $\mu$s & 8.088 $\mu$s & 39.59 ms & 128.14 ms\\ \hline
        Durchnittliche Laufzeit einer Additionsfunktion & 2340.688 $\mu$s & 10.053 $\mu$s &39.59 ms &  29.77 ms \\ \hline
        Durchnittliche Laufzeit von Übergabe eines .NET-Objekts & 9054.007 $\mu$s & 7548.497 $\mu$s &  66.72 ms & 46.27 ms  \\ \hline
    \end{tabular}
\end{table}

\newpage
Es folgen nun die Resultate von BenchmarkDotNet und der ausgeführte Code.\\

Für:

Lua und IronPython
     \begin{table}[H]
            \begin{tabular}{|p{3.5cm}|p{3cm}|p{3cm}|p{3cm}|}
            \hline
                Method &   Mean &   Error & StdDev \\ \hline
                TestIronPython & 137.118 $\mu$s & 7.5278 $\mu$s & 21.959 $\mu$s \\ \hline
                TestIronPythonSum & 2,340.688 $\mu$s & 71.9924 $\mu$s & 208.863 $\mu$s \\ \hline
                TestLua & 8.088 $\mu$s & 0.3702 $\mu$s & 1.020 $\mu$s \\ \hline
                TestLuaSum & 10.053 $\mu$s & 0.4560 $\mu$s & 1.330 $\mu$s \\ \hline
                TestLua-
                PassDotNetObject- 
                AndCallFunction & 7,548.497 $\mu$s & 1,258.6385 $\mu$s & 3,711.124 $\mu$s \\ \hline
                TestIronPython-
                PassDotNetObject- 
                AndCallFunction & 9,054.007 $\mu$s & 471.9551 $\mu$s & 1,346.514 $\mu$s \\ \hline
            \end{tabular}
    \end{table}
\newpage
Der Code für die Methoden von IronPython demonstriert die Integration von IronPython in eine C\#-Anwendung und umfasst Benchmark-Tests für verschiedene Szenarien. Hier ist eine Zusammenfassung des Codes:

Der Code beginnt mit der Initialisierung einer IronPython-Skript-Engine (engine) und definiert dann drei Methoden für verschiedene Benchmark-Szenarien:
\begin{itemize}
    \item IronPythonSimple:
    \begin{itemize}
        \item Diese Methode erstellt einen Skriptbereich (scope) und eine Skriptquelle (source) mithilfe der IronPython-Skript-Engine.
        \item Das IronPython-Skript in der Methode definiert eine Funktion (fun), die den Wert 42 zurückgibt, und ruft diese Funktion dann auf. Das Ergebnis wird in der Konsole ausgegeben.
    \end{itemize}
    \item IronPythonSum:
    \begin{itemize}
        \item  Diese Methode führt ein weiteres IronPython-Skript aus, das eine Summenfunktion definiert und die Summe von 3 und 3 berechnet. Das Ergebnis wird ebenfalls in der Konsole ausgegeben.
    \end{itemize}
    \item IronPythonPassDotNetObjectsAndCallFunction:
    \begin{itemize}
        \item Diese Methode zeigt, wie man ein .NET-Objekt (SomeClass) an ein IronPython-Skript übergibt und eine Funktion (Func1) darauf aufruft.
        Hierzu wird die Methode 'SetPythonScript' verwenden.
        Ein Skriptbereich wird erstellt, das .NET-Objekt wird dem Skriptbereich zugeordnet, und dann wird das Skript ausgeführt. Das Ergebnis wird in der Konsole ausgegeben.
    \end{itemize}
    \item SetPythonScript:
    \begin{itemize}
        \item Diese Methode erstellt ein Python-Skript in Form eines Strings. Das Skript importiert die Common Language Runtime (clr) und ruft die Methode Func1() des übergebenen .NET-Objekts auf.
    \end{itemize}

\end{itemize}
Drei Benchmark-Methoden (TestIronPython, TestIronPythonSum, TestIronPythonPassDotNetObjectsAndCallFunction) sind vorhanden und rufen die entsprechenden IronPython-Methoden auf. Diese Benchmark-Methoden sind mit dem [Benchmark]-Attribut versehen und dienen zur Leistungsmessung.
    

\newpage
 Der Code für die Methoden von NLua:
\begin{lstlisting}[language={[Sharp]C}, caption=NluaTestMethods, label=lst:imp:nluam]
    private readonly Lua state = new Lua();

    
    public void LuaPassDotNetObjectAndCallFunction()
    {
        var obj = new SomeClass();
        state["obj"] = obj;
        state.DoString (@"result=obj:Func1()");
        var result = state["result"];
        Console.WriteLine(result);
    }

    public void LuaSimple()
    {
        state.DoString("function fun() \r\n\t  return 42 \r\n end \r\n test=fun()");
        var test = state["test"];
        Console.WriteLine(test);
    }

    public void LuaSum()
    {


        state.DoString("function sum(x,y) \r\n\t return x+y \r\n end \r\n result=sum(3,3)");
        var result = state["result"];
        Console.WriteLine(result);
    }
    
\end{lstlisting}

\newpage
C\#script
        \begin{table}[H]
            \begin{tabular}{|p{3.5cm}|p{3cm}|p{3cm}|p{3cm}|}
            \hline
                Method & Mean & Error & StdDev \\ \hline
                TestCsharpSimple & 39.59 ms & 0.354 ms & 0.331 ms \\ \hline
                TestCsharpSum & 39.56 ms & 0.321 ms & 0.301 ms \\ \hline
                TestCsharp-
                Objects & 66.72 ms & 0.875 ms & 0.819 ms \\ \hline
            \end{tabular}
        \end{table}

        \begin{lstlisting}[language={[Sharp]C}, caption=\#ScriptingTestMethods, label=lst:imp:cscm]
            public static async void ReturnNumber()
        {
            var state = await CSharp#Script.RunAsync("return 42;");
            Console.WriteLine(state.ReturnValue);
        }
        public static async void MySum()
        {
            var state = await CSharpScript.RunAsync("return 3 + 3;");
            Console.WriteLine(state.ReturnValue);
        }
        public static async void DotNetObject()
        {
            var obj = new Student("Hans", 18);

            var globals = new Globals { student = obj };
            var state = await CSharpScript.RunAsync("", globals: globals);
        }
        \end{lstlisting}
Javascript
     \begin{table}[H]
            \begin{tabular}{|p{3.5cm}|p{3cm}|p{3cm}|p{3cm}|}
            \hline
                Method & Mean & Error & StdDev \\ \hline
                TestJavascriptSimple & 128.14 ms & 1,102.64 ms & 286.35 ms  \\ \hline
                TestJavaScriptSum & 29.77 ms & 255.31 ms & 66.30 ms \\ \hline
                TestJavascript-
                DotNetObjects & 46.27 ms & 397.72 ms & 103.29 ms  \\ \hline
            \end{tabular}
        \end{table}

        \begin{lstlisting}[language={[Sharp]C}, caption=JavascriptTestMethods, label=lst:imp:jsm]
            public void JavascriptSimple()
            {
                var engine = new JintJsEngine();           
                engine.Execute(@"
                                function myFunction() {
                                    return 42;
                                }");           
            }
            public void JavascriptSum()
            {
                var engine = new JintJsEngine();
    
                engine.Execute(@"
                                function mySum(x,y) {
                                    return x+y;
                                }
                                var x = mySum(3,3);");
            }
            public void DotNetObjects()
            {
                var engine = new JintJsEngine();
                var obj = new Student("Hans", 18);
                var student = JsonConvert.SerializeObject(obj);
                engine.SetVariableValue("student", student);           
            }
        \end{lstlisting}


\newpage
\subsection{Funktionalität}
In der nachfolgenden Tabelle sind die Recherche-Ergebnisse hinsichtlich der Funktionalität der Scriptsprachen in .NET dargestellt.
Die Informationen wurden aus den offizellen Webseiten der Nuget-Pakete entnommen.

\begin{table}[H]
    \begin{tabular}{|p{3cm}|p{3cm}|p{3cm}|p{3cm}|p{3cm}|}
        \hline
        Funktion & IronPython & Lua & C\#Scripting & Javascript\\ \hline
        Kann auf .NET Variablen zugreifen & Ja & Ja & Ja & Ja \\ \hline
        Kann globale Variablen & Ja & Ja & Ja & Ja \\ \hline
        Unterstützt Erweiterungspakete der Scriptsprache & Ja (nicht numpy und pandas!) & Ja & Ja & Ja \\ \hline 
    \end{tabular} 
\end{table}
\newpage
\subsection{Debugging}
Das Debugging durch die Ausgabe auf der Konsole ist mit allen, in dieser Diplomarbeit untersuchten, Scriptsprachen möglich.

Breakpoints in VisualStudio bzw. in VisualStudio-Code sind nur mit IronPython möglich.


Es folgen nun Screenshots, um zu beweisen, dass die Angaben stimmen.

\begin{itemize}
    \item Der Beweis für das Debugging mit NLua:
    
    \begin{figure}[H]
        \centering
        \includegraphics[scale=0.5]{pics/Lua-Konsolenausgabe.png}
        \caption{Lua-Konsolenausgabe}
        \label{fig:impl:KonsolenausgabeLua}
    \end{figure}

\newpage
    \item Die Beweise für das Debugging mit IronPython:
    \begin{figure}[H]
        \centering
        \includegraphics[scale=0.5]{pics/IronPythonVSCodeBreakpoint1.png}
        \caption{IronPython-VSCode-Breakpoint1}
        \label{fig:impl:IronPythonVSCodeBreakpoint1}
    \end{figure}

    \begin{figure}[H]
        \centering
        \includegraphics{pics/IronPythonVSBreakpoint2.png}
        \caption{IronPython-VSBreakpoint2}
        \label{fig:impl:IronPythonVSBreakpoint2}
    \end{figure}

    \begin{figure}[H]
        \centering
        \includegraphics[scale=0.5]{pics/IronPythonKonsolenausgabe.png}
        \caption{IronPython-Konsolenausgabe}
        \label{fig:impl:IronPythonKonsolenausgabe}
    \end{figure}
\end{itemize}

\begin{spacing}{1}
\chapter{Anwendung (Praxisteil)}\label{chapter:tech}

\end{spacing}
\section{Verwendete Technologien}
\setauthor{Robert Freiseisen}

Zur Realisierung der Beispielanwendung wurden folgende Technologien verwendet:
Docker

\begin{itemize}
    \item Docker
    \begin{itemize}
        \item Docker ist eine Open-Source-Plattform, die es Entwicklern ermöglicht, Anwendungen in Containern zu erstellen und zu betreiben. Ein Container ist eine eigenständige, isolierte Einheit, die alles enthält, was eine Anwendung zur Ausführung benötigt, einschließlich Code, Laufzeitumgebung, Systembibliotheken und -tools. Docker vereinfacht die Bereitstellung und den Betrieb von Anwendungen, da es eine konsistente Umgebung bietet, die über verschiedene Systeme und Cloud-Dienste hinweg gleich bleibt.
    \end{itemize}

    \item Docker-Compose
    \begin{itemize}
        \item Docker Compose ist ein Tool für die Definition und Verwaltung von Multi-Container-Docker-Anwendungen. Es ermöglicht Entwicklern, eine docker-compose.yml Datei zu verwenden, in der sie die Dienste, Netzwerke und Volumes der Anwendung definieren können. Durch Ausführen eines einzigen docker-compose up Befehls können alle Dienste und Abhängigkeiten in der Reihenfolge gestartet werden, wie sie in der YAML-Datei definiert sind. Das Tool ist besonders nützlich für die Orchestrierung von Microservices und komplexen Anwendungen.
    \end{itemize}
\newpage
    \item ASP.NET
    \begin{itemize}
        \item ASP.NET ist ein Web-Framework von Microsoft, das für die Entwicklung von Webseiten, Webanwendungen und Webdiensten verwendet wird. Es ist in der .NET-Plattform eingebettet und bietet eine Reihe von Bibliotheken und Tools für die schnelle und effiziente Entwicklung. ASP.NET kann mit verschiedenen Programmiersprachen wie Csharp, Fsharp und VB.NET verwendet werden. 
        Es unterstützt sowohl MVC (Model-View-Controller) als auch Web API, was es zu einer vielseitigen Wahl für viele Arten von Webprojekten macht.
    \end{itemize}

    \item AutoMapper
    \begin{itemize}
        \item AutoMapper ist eine Open-Source-Bibliothek für die objekt-orientierte Programmierung, die automatische Zuordnungen zwischen zwei Objekten unterschiedlichen Typs ermöglicht. Es wird oft in Csharp-Projekten und im Kontext von .NET-Anwendungen verwendet. Durch die Verwendung von Konventionen statt der expliziten Konfiguration kann AutoMapper den Boilerplate-Code reduzieren, der normalerweise erforderlich ist, um Daten von einem Datenmodell in ein anderes zu übertragen. Es ist besonders nützlich in Szenarien wie dem Mapping zwischen Datenzugriffsobjekten (DAOs) und Data Transfer Objects (DTOs).
    \end{itemize}
\end{itemize} 

\newpage
\section{Aufbau}
\setauthor{Robert Freiseisen}

Um einen Überblick über die Beispielanwendung zu erhalten folgt nun ein Komponentendiagramm:

\begin{figure}[H]
    \centering
    \includegraphics[scale=0.5]{pics/KomponentenDiagramm.png}
    \caption{Komponenten -- UML Diagramm}
    \label{fig:impl:KomponentenDiagramm}
\end{figure}

\newpage
Damit der Aufbau der .NET-Solution noch klarer wird ist nun die YAML-Datei dargestellt:

\lstinputlisting[style=yaml]{input-files/docker-compose.yml}

\newpage

Die verwendeten Entitäten und ihre Relationen im Backend sind in der folgenden Abbildung dargestellt.

\begin{figure}[H]
    \centering
    \includegraphics[scale=0.5]{pics/EntitiesClassDiagram.png}
    \caption{Entitäten -- UML Diagramm}
    \label{fig:impl:Entities}
\end{figure}

\newpage
Es gibt viele Möglichkeiten Skripte in eine Anwendung zu importieren.
Eine Möglichkeit ist die Skripte als Datei zu importieren.
Anstatt alle benötigten Daten direkt in den Code einzubetten oder sie manuell über die Befehlszeile einzugeben, 
können Entwickler eine oder mehrere Dateien als Input verwenden, die das Skript dann liest und verarbeitet.

\begin{figure}[H]
    \centering
    \includegraphics[scale=0.5]{pics/LogicClassDiagram.png}
    \caption{Logic Overview}
    \label{fig:impl:Logic}
\end{figure}


\newpage
Im folgenden Abschnitt  werden einige Code-Ausschnitte betrachtet, die zeigen, wie man solche Datei-Übergaben in den untersuchten Scriptsprachen realisieren kann. 

\begin{lstlisting}[language={[Sharp]C},caption=Code for Javascript,label=lst:impl:js]
    public class JavascriptRunner
    {
        public Grade RunScript(GradeKey key, List<Grade> grades)
        {
            var engine = new JintJsEngine();
            Grade result = new Grade();
            List<string>? logs = new List<string>();

            try
            {
                // Definiere eine Variable im JavaScript-Code, um die console.log-Ausgaben zu speichern
                engine.Execute("var consoleOutput = [];");

                // Definiere die console.log-Funktion im JavaScript-Code
                engine.Execute(@"
                                var console = {
                                    log: function() {
                                        consoleOutput.push(Array.from(arguments).join(' '));
                                    }
                                };
                            ");


                var gradeKindsList = JsonConvert.SerializeObject(key.UsedKinds);
                var gradesList = JsonConvert.SerializeObject(grades);

                engine.SetVariableValue("gradeKindsList", gradeKindsList);
                engine.SetVariableValue("gradesList", gradesList);

                if (key.Calculation != null)
                {
                    engine.Execute(key.Calculation);
                }

                //Die Ausgabe der console.log-Anweisungen als JSON-String
                string jsonOutput = engine.Evaluate<string>("JSON.stringify(consoleOutput)");

                // Konvertiere den JSON-String in eine Liste von strings
                if (jsonOutput != null)
                {
                    logs = JsonConvert.DeserializeObject<List<string>>(jsonOutput);

                    if (logs != null)
                    {
                        DisplayOutput(logs);   
                    }
                }

                // Get Return from Script
                var resultGrade = engine.GetVariableValue("result");

                result.Teacher = key.Teacher;
                result.Graduate = Convert.ToInt32(resultGrade);
            }
            catch (Exception)
            {
                result.Teacher = null;
                result.Graduate = 0;
            }
            return result;
        }

        private static void DisplayOutput(List<string> logs)
        {
            foreach (string output in logs)
            {
                Debug.WriteLine(output);
            }
        }
    }
\end{lstlisting}


\begin{lstlisting}[language={[Sharp]C},caption=Code for NLua,label=lst:impl:nlua]
    /// <summary>
    /// Runs lua-scripts
    /// </summary>
    public class LuaScriptRunner
    {
        private readonly Lua state;

        public LuaScriptRunner()
        {
            this.state = new Lua();
        }

        public Grade RunScript(GradeKey key, List<Grade> grades)
        {
            if (key.Calculation == string.Empty || key.UsedKinds == null || grades == null)
            {
                throw new NullReferenceException("Not enough information for Calculation");
            }

            var code = key.Calculation;

            var result = new Grade();
            try
            {
                state.DoString(code);
                state.LoadCLRPackage();
                state["grades"] = grades;
                state.DoString(@"graduate = calculate()");
                result.Teacher = key.Teacher;
                var gr = state["graduate"];
                if (gr != null) 
                {
                    result.Graduate = Convert.ToInt32(gr);
                }
            }
            catch (Exception)
            {
                throw;
            }

            return result;
        }
    }
\end{lstlisting}

\newpage

\begin{lstlisting}[language={[Sharp]C},caption=Code for CsharpScripting,label=lst:impl:csc]
    public class CsScriptRunner
    {
        public static Grade RunScript(GradeKey key, List<Grade> grades)
        {
            var result = new Grade();

            // StringWriter erstellen, um die Ausgabe des Skripts zu erfassen
            StringWriter sw = new StringWriter();

            // Console.Out umleiten
            TextWriter originalOut = Console.Out;
            Console.SetOut(sw);
            try
            {                
                dynamic script = CSScript.Evaluator
                    .ReferenceAssemblyOf(typeof(GradeKey))
                    .ReferenceAssemblyOf(typeof(Grade))
                    .CompileCode(key.Calculation)
                    .CreateObject("*");
                
                var res = script.Calculate(key, grades);
                result.Teacher = key.Teacher;              
                result.Graduate = Convert.ToInt32(res);

            }
            catch (Exception)
            {
                result.Teacher = null;
                result.Graduate = 0;
            }
            finally
            {
                // Output ausgeben
                Debug.WriteLine(sw);
            }

            return result;
        }
    }
\end{lstlisting}




\begin{spacing}{1}
\chapter{Technologien}\label{chapter:tech}
\end{spacing} 

\begin{spacing}{1}
\chapter{Umsetzung}\label{chapter:implementation}
\end{spacing}
Siehe tolle Daten in Tab. \ref{tab:impl:data}.

\begin{table}
    \centering
    \begin{tabular}{|lcc|}
    \hline
              & \textbf{Regular Customers} & \textbf{Random Customers} \\ \hline
    Age       & 20-40                      & \textgreater{}60          \\ \hline
    Education & university                 & high school               \\ \hline
    \end{tabular}
    \caption{Ein paar tabellarische Daten}
    \label{tab:impl:data}
\end{table}

\begin{figure}
    \centering
    \includegraphics[scale=0.5]{pics/knuthi.jpg}
    \caption{Don Knuth -- CS Allfather}
    \label{fig:impl:knuth}
\end{figure}

Siehe und staune in Abb. \ref{fig:impl:knuth}.
\lipsum[6-9]
Dann betrachte den Code in Listing \ref{lst:impl:foo}.



\begin{spacing}{1}
\chapter{Zusammenfassung}
\end{spacing}
\section{Schlussfolgerungen}
\setauthor{Philipp Füreder}

Diese Diplomarbeit hat sich mit der Möglichkeit von Scripting in .NET-Anwendungen zur 
Laufzeit auseinandergesetzt. Ziel der Arbeit war es, die Möglichkeiten und Grenzen dieser 
Technologie sowohl aus Entwickler/in- als auch aus als Benutzer/in zu untersuchen.

Entwickler/innen können die Erkenntnisse nutzen, um anpassungsfähigere .NET-Anwendungen zu bauen. 
Zum Beispiel können sie Scripting einsetzen, um fehlende Funktionen von Anwendungen zu erstellen, 
ohne viele Codeänderungen zu machen. Die Untersuchung des Scripting in .NET-Anwendungen zur 
Laufzeit hat eine Reihe von wichtigen Erkenntnissen geliefert. 
Diese können dazu beitragen, die theoretischen Grundlagen in diesem Bereich zu erweitern 
und gleichzeitig praxisorientierte Lösungen für die Softwareentwicklung und -sicherheit zu bieten.
Die Arbeit legt somit einen wichtigen Grundstein für weiterführende 
Untersuchungen und Entwicklungen in diesem Bereich.\\

Während das Scripting in .NET-Anwendungen zur Laufzeit eine leistungsstarke Funktion für die 
dynamische Modifikation und Anpassung darstellt, hat es einen wesentlichen Nachteil: 
das Debugging des zur Laufzeit integrierten Skripts ist nicht möglich. Diese Limitation 
stellt eine Herausforderung für Entwickler/innen dar. Das bedeutet, dass zwar von der Flexibilität 
des Scripting profitiert werden kann, jedoch Schwierigkeiten Fehler im Code effizient zu 
identifizieren und zu beheben. Unsere Alternative für das Debugging ist die Konsolen-Ausgabe. 
Obwohl diese Methode weniger umfassend ist als Debugging-Tools, ermöglicht sie dennoch eine 
gewisse Überwachung und Fehleridentifikation in Echtzeit. Sie erlaubt es Entwickler/innen, 
wichtige Informationen, Zustände oder Fehlermeldungen direkt in der Konsole auszugeben, 
um so das Verhalten des Skripts zur Laufzeit besser nachvollziehen zu können.

\newpage
\section{Herausforderungen und Probleme}
\setauthor{Philipp Füreder}

Im Verlauf unserer Diplomarbeit über Scripting in .NET-Anwendungen zur Laufzeit sind wir 
auf mehrere Herausforderungen gestoßen, die unsere Arbeit zunächst erschwert haben, 
uns aber letztlich zu wichtigen Erkenntnissen geführt haben.

\subsection*{Verständnis der Problemstellung}
Einer der ersten Stolpersteine war das grundlegende Verständnis der Problemstellung. 
Scripting in .NET-Anwendungen zur Laufzeit ist ein komplexes Thema, das sowohl technisches 
als auch konzeptionelles Verständnis erfordert. Es dauerte einige Zeit, bis wir die 
Kernprobleme und -fragen unserer Forschung vollständig erfassen konnten.

\subsection*{Overengineering}
Eine unserer größten Herausforderungen bestand darin, dass wir die Anwendung in der 
Anfangsphase übermäßig komplex gestaltet hatten. Dieses Over Engineering führte zu 
unerwarteten Problemen und machte eine Korrektur notwendig. Nach einer kritischen 
Überprüfung unseres Ansatzes entschieden wir, mehrere Komponenten zu entfernen, 
um die Anwendung zu vereinfachen und den Fokus auf die Kernthemen zu legen.

\subsection*{Implementierung der Konsolenausgabe}
Die Konsolenausgabe erschien zunächst als einfache Lösung für unsere Debugging-Anforderungen. 
Allerdings stellte die tatsächliche Implementierung eine größere Herausforderung dar als erwartet,
insbesondere in Bezug auf die Synchronisierung der Konsolenausgabe mit den zur Laufzeit 
generierten Skripten.

\newpage
\subsection*{Fazit}
Diese anfänglichen Herausforderungen haben unseren Lernprozess und unsere methodische 
Herangehensweise wesentlich geprägt. Jedes dieser Probleme wurde überwunden, aber die 
dabei gesammelten Erfahrungen haben wesentlich zu unserer fachlichen Entwicklung 
beigetragen und werden zweifellos von Nutzen sein, wenn wir uns zukünftigen 
Forschungsprojekten stellen.

\newpage
\section{Kritische Betrachtung der Ergebnisse}
\setauthor{Philipp Füreder}

Während unsere Diplomarbeit wichtige Einblicke in die Möglichkeiten und Herausforderungen 
des Scripting in .NET-Anwendungen zur Laufzeit liefert, muss eine kritische Betrachtung 
unserer Ergebnisse einige Einschränkungen berücksichtigen. Erstens ist unser Forschungsumfeld 
auf eine begrenzte Anzahl von Skriptsprachen und Anwendungsbeispielen beschränkt, 
was die Allgemeingültigkeit der Erkenntnisse einschränken könnte. Zweitens basiert 
unsere alternative Debugging-Methode über die Konsolenausgabe auf einer vereinfachten 
Annahme des Systemverhaltens, die in komplexeren oder sicherheitskritischen Anwendungen 
nicht ausreichend sein könnte. 

\newpage
\section{Mögliche weitere Untersuchungsthemen}
\setauthor{Philipp Füreder}

Die Untersuchung von Scripting in .NET-Anwendungen zur Laufzeit stellt nur die Spitze des 
Eisbergs dar, wenn es um die Komplexität und Vielfältigkeit des Themenfelds geht. 
Die Ergebnisse der vorliegenden Arbeit legen zahlreiche Ansatzpunkte für zukünftige 
Forschungsprojekte nahe.

\subsection*{Sicherheitsaspekte}

Zukünftige Studien könnten spezifische Angriffsszenarien und ihre Abwehrmöglichkeiten analysieren.
Besonders die sich ständig weiterentwickelnde Landschaft von Sicherheitsbedrohungen 
stellt einen fruchtbaren Boden für weiterführende Untersuchungen dar.

\subsection*{Performance-Optimierung}

Ein weiteres interessantes Forschungsfeld könnte die Performance-Optimierung von 
.NET-Anwendungen sein, die intensiv Scripting zur Laufzeit nutzen. Hier könnte 
untersucht werden, wie sich verschiedene Scripting-Techniken auf die Laufzeitleistung 
der Anwendung auswirken und wie sich diese Performance am besten optimieren lässt.

\subsection*{Erweiterte Debugging-Methoden}

Angesichts der Schwierigkeiten beim Debugging zur Laufzeit wäre es lohnend, 
innovative Methoden oder Tools für diese spezifische Herausforderung zu entwickeln 
und zu evaluieren. Wie können Entwickler und Sicherheitsexperten noch effektiver das 
Verhalten von Skripten in Echtzeit nachvollziehen?

\newpage
\pagenumbering{Roman}
\setcounter{page}{\value{RPages}}
\input{glossary}
%\setlength{\glsdescwidth}{0.8\linewidth}
\glsnogroupskiptrue
\printglossary[title=Glossar,toctitle=Glossar] %,style=long]
\spacing{1}{
%\bibliographystyle{IEEEtran}
\bibliographystyle{ieeetrande}
\bibliography{bib}
}
\listoffigures
\listoftables
\lstlistoflistings
\appendix
\addchap{Anhang}
\input{./sections/appendix}
\end{document}

