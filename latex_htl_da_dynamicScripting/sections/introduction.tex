\section {Standardsoftware vs Individualsoftware}
\setauthor{Philipp Füreder}

In der heutigen digitalisierten Welt ist Software enorm wichtig für den Erfolg von Unternehmen 
oder auch für den privaten Gebrauch.
Es gibt zwei Hauptarten von Software, die Unternehmen nutzen: \textbf{Standardsoftware} und 
\textbf{Individualsoftware}.
Beide haben Vor- und Nachteile, und es ist wichtig, 
diese gut zu verstehen, um die richtige Wahl für das eigene Unternehmen zu treffen.

\subsection*{Standardsoftware}

Standardsoftware ist üblicherweise sehr Benutzerfreundlich und einfach zu verwenden. 
Sie wird meist von großen Unternehmen entwickelt 
(Beispiel: Microsoft Office, Adobe Creative Cloud oder auch Google Workspace) 
und ist in der Anschaffung billiger als Individualsoftware. 
Der Grund dafür ist, dass Standardsoftware für die breite Masse entwickelt wird 
und für jeden Nutzer die gleichen Funktionen besitzt. 
Außerdem sind sie in der Regel weiter verbreitet und leichter zu erwerben.

Ein Nachteil von Standardsoftware besteht darin, dass sie nicht immer genau 
den individuellen Anforderungen eines Unternehmens gerecht wird. 
Es könnte vorkommen, dass spezifische Funktionen fehlen oder die Software 
nicht optimal auf die Arbeitsprozesse des Unternehmens zugeschnitten ist. 

\subsection*{Individualsoftware}

Auf der anderen Seite wird Individualsoftware speziell für ein einzelnes 
Unternehmen oder eine bestimmte Aufgabe entworfen. 
Das bedeutet, es ist eine individuelle Lösung, die exakt auf 
die Anforderungen und Bedürfnisse der Benutzer/innen zugeschnitten ist.

Die Verwendung von Individualsoftware hat aber auch ihre Nachteile. 
Zum einen ist sie in der Regel kostenintensiver, da die individuelle Entwicklung 
mehr Ressourcen erfordert. Zum anderen gestaltet sich die Umsetzung und Wartung 
anspruchsvoller, da die Einzigartigkeit der Software spezifische Herausforderungen 
mit sich bringt, die über die Zeit hinweg bewältigt werden müssen.

\section{Problemverständnis}

Die meisten Benutzer/innen sind mit den gängigsten Standardsoftwares bereits vertraut. 
Deshalb wäre eine neue Individualsoftware eine Umstellung, wo man wieder Zeit aufwenden muss, 
um sich damit zurecht zu finden. Und außerdem muss man diese dann ja auch implementieren, 
warten und vor allem finanzieren, was sehr kostspielig sein kann. 
\newline
Es mag verlockend sein, Standardsoftware mit den benötigten Funktionen für jeden Kunden anzupassen. 
Aber das bringt viele Probleme mit sich. Wenn man für jeden Kunden 
spezielle Funktionen einbaut, muss man den Software-Code jedes Mal stark ändern. 
Das macht die Software kompliziert und schwer zu pflegen. Es können auch Fehler auftreten, 
wenn neue Funktionen hinzugefügt werden, und es wird schwierig, die Software auf dem 
neuesten Stand zu halten.

\newpage
\section{Lösungsansätze: Scripting und Alternativen}
\setauthor{Philipp Füreder}

Ein vielversprechender Ansatz zur Lösung von fehlenden Funktionen in einer Software besteht darin 
Scripting zu nutzen. Dadurch erhalten Benutzer/innen die Möglichkeit, eigene Scripts zu erstellen, 
um spezifische Funktionen innerhalb der Software auszuführen. Diese Scripts werden 
zur Laufzeit in die Anwendung geladen.

Scripting ermöglicht es individuelle Lösungen zu erstellen, 
die den eigenen Anforderungen gerecht werden. 
Dies erweitert den Nutzen der Software, ohne auf offizielle Updates oder neue Versionen 
warten zu müssen oder sogar eine eigene Software entwickeln zu müssen.

\subsection*{Allgemeines zu Scripting}

Eine Skriptsprache wird vor allem verwendet, um Websites und Webanwendungen zu erstellen. 
Wenn man ein Skript schreibt, baut man kein völlig neues Programm von Grund auf. 
Stattdessen verknüpft man bestehende Teile eines Programms miteinander. 
Dann führt das Programm dieses Skript aus.
Skripte automatisieren Aufgaben. Im Gegensatz zu anderen Programmiersprachen sind 
Skriptsprachen einfacher und oft leichter zu lernen.


\subsection*{Vorteile}

\subsection*{Nachteile}

\subsection*{Alternativen}

\subsection*{Levels von Scripting}







