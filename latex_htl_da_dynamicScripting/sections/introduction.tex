\section {Standardsoftware vs Individualsoftware}
\setauthor{Philipp Füreder}

In der heutigen digitalisierten Welt ist Software enorm wichtig für den Erfolg von Unternehmen 
oder auch für den privaten Gebrauch.
Es gibt zwei Hauptarten von Software, die Unternehmen nutzen: \textbf{Standardsoftware} und 
\textbf{Individualsoftware}.
Beide haben Vor- und Nachteile, und es ist wichtig, 
diese gut zu verstehen, um die richtige Wahl für das eigene Unternehmen zu treffen.

\subsection*{Standardsoftware}

Standardsoftware ist üblicherweise sehr Benutzerfreundlich und einfach zu verwenden. 
Sie wird meist von großen Unternehmen entwickelt 
(Beispiel: Microsoft Office, Adobe Creative Cloud oder auch Google Workspace) 
und ist in der Anschaffung billiger als Individualsoftware. 
Der Grund dafür ist, dass Standardsoftware für die breite Masse entwickelt wird 
und für jeden Nutzer die gleichen Funktionen besitzt. 
Außerdem sind sie in der Regel weiter verbreitet und leichter zu erwerben.

Ein Nachteil von Standardsoftware besteht darin, dass sie nicht immer genau 
den individuellen Anforderungen eines Unternehmens gerecht wird. 
Es könnte vorkommen, dass spezifische Funktionen fehlen oder die Software 
nicht optimal auf die Arbeitsprozesse des Unternehmens zugeschnitten ist. 

\subsection*{Individualsoftware}

Auf der anderen Seite wird Individualsoftware speziell für ein einzelnes 
Unternehmen oder eine bestimmte Aufgabe entworfen. 
Das bedeutet, es ist eine individuelle Lösung, die exakt auf 
die Anforderungen und Bedürfnisse der Benutzer/innen zugeschnitten ist.

Hingegen hat die Verwendung von Individualsoftware auch ihre Nachteile. 
Zum einen ist sie in der Regel kostenintensiver, da die individuelle Entwicklung 
mehr Ressourcen erfordert. Zum anderen gestaltet sich die Umsetzung und Wartung 
anspruchsvoller, da die Einzigartigkeit der Software spezifische Herausforderungen 
mit sich bringt, die über die Zeit hinweg bewältigt werden müssen.

