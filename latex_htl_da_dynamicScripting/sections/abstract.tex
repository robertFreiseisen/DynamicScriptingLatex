\begin{spacing}{1}
    \chapter*{Abstract}
\end{spacing}
\begin{wrapfigure}{r}{0.3\textwidth}
    \begin{center}
      \includegraphics[width=0.3\textwidth]{pics/coding.jpg}
    \end{center}
\end{wrapfigure}
This diploma thesis is dedicated to the extension of standard software through the targeted use of 
scripting. The focus is on the development of flexible mechanisms that make it possible to adapt 
and extend existing software to individual requirements using scripts. The work examines the 
applicability of this method in various contexts and evaluates its effectiveness and adaptability.
\\\\
We demonstrate the practical implementation of these concepts using an example application. The 
development of this application demonstrates not only the theoretical feasibility but also the 
practical suitability of the approach. The investigation focuses on the efficiency and scalability 
of the scripting solutions, analyzing resource consumption and runtime to enable a well-founded 
assessment of their performance.

\newpage
\begin{spacing}{1}
    \chapter*{Zusammenfassung}
\end{spacing}
\begin{wrapfigure}{r}{0.3\textwidth}
    \begin{center}
      \includegraphics[width=0.3\textwidth]{pics/coding.jpg}
    \end{center}
\end{wrapfigure}
Diese Diplomarbeit widmet sich der Erweiterung von Standardsoftware durch den gezielten Einsatz 
von Scripting. Der Fokus liegt auf der Entwicklung flexibler Mechanismen, die es ermöglichen, 
bestehende Software durch den Einsatz von Skripten an individuelle Anforderungen anzupassen und 
zu erweitern. Die Arbeit untersucht die Anwendbarkeit dieser Methode in verschiedenen Kontexten 
und evaluiert ihre Wirksamkeit sowie Anpassungsfähigkeit.
\\\\
Die praktische Umsetzung dieser Konzepte zeigen wir anhand einer Beispielanwendung. Durch die 
Entwicklung dieser Anwendung wird nicht nur die theoretische Machbarkeit, sondern auch die 
Praxistauglichkeit des Ansatzes demonstriert. Die Untersuchung konzentriert sich auf die Effizienz 
und Skalierbarkeit der Scripting-Lösungen, analysiert den Ressourcenverbrauch und die Laufzeit, 
um eine fundierte Bewertung ihrer Performance zu ermöglichen.
