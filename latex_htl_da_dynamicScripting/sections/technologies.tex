\section{Verwendete Technologien}
\setauthor{Philipp Füreder}

\subsection*{ASP .NET Core}

Unsere Webanwendung wurde unter Verwendung der ASP.NET Core-Plattform entwickelt. 
ASP.NET Core ist ein vielseitiges, plattformübergreifendes und leistungsfähiges 
Open-Source-Framework, das zur Entwicklung moderner, internetfähigen Anwendungen geeignet ist. 
Die Ausführung findet in der .NET Core-Laufzeitumgebung statt.
ASP.NET Core bietet außerdem eine moderne und flexible Umgebung für die Entwicklung von Webanwendungen, 
die sowohl plattformübergreifend als auch hochgradig skalierbar sind.

Mit ASP.NET Core kann man:

\begin{itemize}

\item Webanwendungen und Webdienste, Internet-der-Dinge (IoT)-Anwendungen und 
mobile Backends entwickeln.
\item Auf verschiedene Betriebssysteme wie Windows, macOS und Linux arbeiten.
\item Anwendungen sowohl in der Cloud als auch auf lokalen Systemen bereitstellen.
\end{itemize}

Dieses Framework eröffnet somit eine breite Palette an Möglichkeiten für Entwickler, 
um moderne Anwendungen zu erstellen, die sich nahtlos mit dem Internet verbinden und 
sowohl in Cloud- als auch lokalen Umgebungen effizient betrieben werden können.
\newpage

\subsection*{Git}

Unsere Versionskontrolle haben wir mit Git gemacht. Git ist ein Versionskontrollsystem 
mit verteiltem Ansatz, das entworfen wurde, um sowohl kleine als auch äußerst 
umfangreiche Projekte auf schnelle und effiziente Weise zu verwalten.
Die Erlernbarkeit von Git gestaltet sich einfach, und seine geringe Systembelastung 
geht einher mit herausragender Performance. Es setzt sich von anderen Versionskontrollsystemen 
wie Subversion, CVS, Perforce und ClearCase ab, indem es Funktionen wie kosteneffiziente 
lokale „Branches“, bequeme Staging-Bereiche und vielfältige Arbeitsabläufe bietet.


\subsection*{Docker}



\subsection*{Benchmark}



\subsection*{Bogus}