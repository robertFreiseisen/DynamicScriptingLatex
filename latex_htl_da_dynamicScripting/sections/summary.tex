\section{Schlussfolgerungen}
\setauthor{Philipp Füreder}

Diese Diplomarbeit hat sich mit der Möglichkeit von Scripting in .NET-Anwendungen zur 
Laufzeit auseinandergesetzt. Ziel der Arbeit war es, die Möglichkeiten und Grenzen dieser 
Technologie sowohl aus Entwickler/in- als auch aus als Benutzer/in zu untersuchen.

Entwickler/innen können die Erkenntnisse nutzen, um anpassungsfähigere .NET-Anwendungen zu bauen. 
Zum Beispiel können sie Scripting einsetzen, um fehlende Funktionen von Anwendungen zu erstellen, 
ohne viele Codeänderungen zu machen. Die Untersuchung des Scripting in .NET-Anwendungen zur 
Laufzeit hat eine Reihe von wichtigen Erkenntnissen geliefert. 
Diese können dazu beitragen, die theoretischen Grundlagen in diesem Bereich zu erweitern 
und gleichzeitig praxisorientierte Lösungen für die Softwareentwicklung und -sicherheit zu bieten.
Die Arbeit legt somit einen wichtigen Grundstein für weiterführende 
Untersuchungen und Entwicklungen in diesem Bereich.\\

Während das Scripting in .NET-Anwendungen zur Laufzeit eine leistungsstarke Funktion für die 
dynamische Modifikation und Anpassung darstellt, hat es einen wesentlichen Nachteil: 
das Debugging des zur Laufzeit integrierten Skripts ist nicht möglich. Diese Limitation 
stellt eine Herausforderung für Entwickler/innen dar. Das bedeutet, dass zwar von der Flexibilität 
des Scripting profitiert werden kann, jedoch Schwierigkeiten Fehler im Code effizient zu 
identifizieren und zu beheben. Unsere Alternative für das Debugging ist die Konsolen-Ausgabe. 
Obwohl diese Methode weniger umfassend ist als Debugging-Tools, ermöglicht sie dennoch eine 
gewisse Überwachung und Fehleridentifikation in Echtzeit. Sie erlaubt es Entwickler/innen, 
wichtige Informationen, Zustände oder Fehlermeldungen direkt in der Konsole auszugeben, 
um so das Verhalten des Skripts zur Laufzeit besser nachvollziehen zu können.

\newpage
\section{Herausforderungen und Probleme}
\setauthor{Philipp Füreder}

Im Verlauf unserer Diplomarbeit über Scripting in .NET-Anwendungen zur Laufzeit sind wir 
auf mehrere Herausforderungen gestoßen, die unsere Arbeit zunächst erschwert haben, 
uns aber letztlich zu wichtigen Erkenntnissen geführt haben.

\subsection*{Verständnis der Problemstellung}
Einer der ersten Stolpersteine war das grundlegende Verständnis der Problemstellung. 
Scripting in .NET-Anwendungen zur Laufzeit ist ein komplexes Thema, das sowohl technisches 
als auch konzeptionelles Verständnis erfordert. Es dauerte einige Zeit, bis wir die 
Kernprobleme und -fragen unserer Forschung vollständig erfassen konnten.

\subsection*{Overengineering}
Eine unserer größten Herausforderungen bestand darin, dass wir die Anwendung in der 
Anfangsphase übermäßig komplex gestaltet hatten. Dieses Over Engineering führte zu 
unerwarteten Problemen und machte eine Korrektur notwendig. Nach einer kritischen 
Überprüfung unseres Ansatzes entschieden wir, mehrere Komponenten zu entfernen, 
um die Anwendung zu vereinfachen und den Fokus auf die Kernthemen zu legen.

\subsection*{Implementierung der Konsolenausgabe}
Die Konsolenausgabe erschien zunächst als einfache Lösung für unsere Debugging-Anforderungen. 
Allerdings stellte die tatsächliche Implementierung eine größere Herausforderung dar als erwartet,
insbesondere in Bezug auf die Synchronisierung der Konsolenausgabe mit den zur Laufzeit 
generierten Skripten.

\newpage
\subsection*{Fazit}
Diese anfänglichen Herausforderungen haben unseren Lernprozess und unsere methodische 
Herangehensweise wesentlich geprägt. Jedes dieser Probleme wurde überwunden, aber die 
dabei gesammelten Erfahrungen haben wesentlich zu unserer fachlichen Entwicklung 
beigetragen und werden zweifellos von Nutzen sein, wenn wir uns zukünftigen 
Forschungsprojekten stellen.

\newpage
\section{Kritische Betrachtung der Ergebnisse}
\setauthor{Philipp Füreder}

Während unsere Diplomarbeit wichtige Einblicke in die Möglichkeiten und Herausforderungen 
des Scripting in .NET-Anwendungen zur Laufzeit liefert, muss eine kritische Betrachtung 
unserer Ergebnisse einige Einschränkungen berücksichtigen. Erstens ist unser Forschungsumfeld 
auf eine begrenzte Anzahl von Skriptsprachen und Anwendungsbeispielen beschränkt, 
was die Allgemeingültigkeit der Erkenntnisse einschränken könnte. Zweitens basiert 
unsere alternative Debugging-Methode über die Konsolenausgabe auf einer vereinfachten 
Annahme des Systemverhaltens, die in komplexeren oder sicherheitskritischen Anwendungen 
nicht ausreichend sein könnte. 

\newpage
\section{Mögliche weitere Untersuchungsthemen}
\setauthor{Philipp Füreder}

Die Untersuchung von Scripting in .NET-Anwendungen zur Laufzeit stellt nur die Spitze des 
Eisbergs dar, wenn es um die Komplexität und Vielfältigkeit des Themenfelds geht. 
Die Ergebnisse der vorliegenden Arbeit legen zahlreiche Ansatzpunkte für zukünftige 
Forschungsprojekte nahe.

\subsection*{Sicherheitsaspekte}

Zukünftige Studien könnten spezifische Angriffsszenarien und ihre Abwehrmöglichkeiten analysieren.
Besonders die sich ständig weiterentwickelnde Landschaft von Sicherheitsbedrohungen 
stellt einen fruchtbaren Boden für weiterführende Untersuchungen dar.

\subsection*{Performance-Optimierung}

Ein weiteres interessantes Forschungsfeld könnte die Performance-Optimierung von 
.NET-Anwendungen sein, die intensiv Scripting zur Laufzeit nutzen. Hier könnte 
untersucht werden, wie sich verschiedene Scripting-Techniken auf die Laufzeitleistung 
der Anwendung auswirken und wie sich diese Performance am besten optimieren lässt.

\subsection*{Erweiterte Debugging-Methoden}

Angesichts der Schwierigkeiten beim Debugging zur Laufzeit wäre es lohnend, 
innovative Methoden oder Tools für diese spezifische Herausforderung zu entwickeln 
und zu evaluieren. Wie können Entwickler und Sicherheitsexperten noch effektiver das 
Verhalten von Skripten in Echtzeit nachvollziehen?